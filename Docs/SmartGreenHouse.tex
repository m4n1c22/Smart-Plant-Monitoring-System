\documentclass[10pt]{article}
\usepackage[utf8]{inputenc}
\usepackage[english]{babel}

\usepackage[T1]{fontenc}
%\usepackage{lipsum} 
 
\usepackage{multicol}
\usepackage{amsmath}
\usepackage{graphicx}

\begin{document}
\title{\textbf{Smart Plant Monitoring System}}
\author{
\quad Sreeram Sadasivam \quad Vishwanath Vadhri \quad Christoph Peusens\\
\quad Supradha Ramesh \quad Mallikarjun Nutti\\
FB 20 Informatik\\
Technische Universität Darmstadt\\
\emph{\{sreeram.sadasivam,viswanath.vadhri,christoph.peusens,}\\
\emph{supradha.ramesh,mallikarjun.nuti\}@stud.tu-darmstadt.de}\\
\date{}
}
\maketitle
\textbf{\abstractname{
\emph {\textbf{\\
%ADD the abstract here...
Plant monitoring is seen as one of the most important tasks in any farming or agriculture based environment. With the inception of Ambient Intelligent systems, there have been a rise in ambient intelligent based devices - Smart Homes\cite{ref001} and other similar technologies involving RFID has evolved over the past few years\cite{ref002}. Integration of such an ambient intelligent system with plant monitoring makes farming easier. In this paper, we discuss about the implementation of a smart plant monitoring system which makes use of the concept ambient intelligence with the use of .Net Gadgeteer which, proactively handles the plant monitoring system. The given implementation works along with a cloud based server and a mobile based device (ideally Android/iOS device) which helps the user to control and see the status of the plant which being monitored by the hardware device. The given circuitry detects changes in the moisture, temperature and light conditions in and around the plant, and performs a machine based curation on the plant by providing necessary irrigation and illumination for the plant. Machine curation is also integrated with active weather forecasting systems which are deployed in the cloud based server using which advanced machine curation is performed. For user based curation, the Android device provides user an option to override a machine curated operation. %conclusion statement to be added...
}}}}

\section*{Motivation}

%Add the motivation part here...

\section*{Related Works}

%Add some related work here...


\section*{Design}

%Add design here...
The proposed system consists of two curation systems. The first one being the \emph{machine based curation} and other being the \emph{user based curation}.

\section*{Evaluation}

%Add evaluation here...

\section*{Inference}

%Add inference here...

\section*{Acknowledgment}

The authors would like to thank Florian Müller of Tele-Kooperation(TK) from Informatik department of Technische Universität Darmstadt for providing the opportunity to work in the selected topic.


%Bibliography....
\begin{thebibliography}{1}

\bibitem{ref001}
Kidd, CoryD. et,al. \emph{``The Aware Home: A Living Laboratory for Ubiquitous Computing Research''}.\hskip 1em plus
  0.5em minus 0.4em\relax  Springer Berlin Heidelberg (1999).

\bibitem{ref002}
Antonio J. Jara, Miguel A. Zamora, and Antonio F. Skarmeta.  \emph{``An internet of things---based personal device for diabetes therapy management in ambient assisted living (AAL).''}.\hskip 1em plus
  0.5em minus 0.4em\relax Personal Ubiquitous Comput. 15, 4 (April 2011), 431-440.  

\end{thebibliography}
\end{document}

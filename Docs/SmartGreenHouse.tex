\documentclass[10pt]{article}
\usepackage[utf8]{inputenc}
\usepackage[english]{babel}

\usepackage[T1]{fontenc}
\usepackage{amsmath}
\usepackage{graphicx}

\begin{document}
\title{\textbf{Smart Plant Monitoring System}}
\author{
\quad Sreeram Sadasivam \quad Vishwanath Vadhri \quad Christoph Peusens\\
\quad Supradha Ramesh \quad Mallikarjun Nutti\\
FB 20 Informatik\\
Technische Universität Darmstadt\\
\emph{\{sreeram.sadasivam,viswanath.vadhri,christoph.peusens,}\\
\emph{supradha.ramesh,mallikarjun.nuti\}@stud.tu-darmstadt.de}\\
\date{}
}
\maketitle
\textbf{\abstractname{
\emph {\textbf{\\
%ADD the abstract here...
Plant monitoring is seen as one of the most important tasks in any farming or agriculture based environment. With the inception of Ambient Intelligent systems, there have been a rise in ambient intelligent based devices - Smart Homes\cite{ref001} and other similar technologies involving RFID has evolved over the past few years\cite{ref002}. Integration of such an ambient intelligent system with plant monitoring makes farming easier. In this paper, we discuss about the implementation of a smart plant monitoring system which makes use of the concept ambient intelligence with the use of .Net Gadgeteer which, proactively handles the plant monitoring system. The given implementation works along with a cloud based server and a mobile based device (ideally Android/iOS device) which helps the user to control and see the status of the plant which is being monitored by the hardware device. The given circuitry detects changes in the moisture, temperature and light conditions in and around the plant, and performs a machine based curation on the plant by providing necessary irrigation and illumination for the plant. Machine curation is also integrated with active weather forecasting systems which are deployed in the cloud based server using which advanced machine curation is performed. For user based curation, the Android device provides user an option to override a machine curated operation. %conclusion statement to be added...
}}}}

\section*{Motivation}

%Add the motivation part here...
The most important factors for the quality and productivity of plant growth are temperature, humidity, light and the level of the carbon dioxide. Continuous monitoring of these environmental variables gives information to the grower to better understand, how each factor affects growth and how to manage maximal growth of plants.Climate control and monitoring of the plant is one of important aspects in agriculture.The aspects we are presenting resembles the concept of precision agriculture, this is the trend of farming presented in recent years for commercial and research agriculture. In the precision agriculture framework the focus is mainly on understanding the environment through the interpretation of a wide variety of data coming from GPS systems, satellite imaging, and in-field sensors. The main motivation of the project is for the user to monitor the plants or cultivation to get enough resources such as light and water without the user need to be present at the plants or cultivation area, and also could manipulate the resources provided to the plants depending on the climate of the plant’s location(eg : if there is enough rain or moisture for the day, the sensor in the soil would detect accordingly for the operation to begin and intimate the user). This could help user not only to give the resources to the plants everyday without much manual effort also helps the constant and healthy growth of a plant.


\section*{Related Works}

%Add some related work here...


\section*{Design}

%Add design here...
The proposed system consists of two curation systems. The first one being the \emph{machine based curation} and other being the \emph{user based curation}. These curative systems are in place to provide a smart and involuntary feedback to the given environment. The former is more of predictive system devised on a hardware whereas, the latter is for the user to actively handle the response provided by hardware. The curation handled by the device are based on the sensor information it receives from its end points(\emph{sensory hardware units}), such a curation is called as \emph{``Direct Machine Based Curation''}. There is also a hybrid machine based curation provided by the device and the cloud server which relies on the Weather Forecast information with the collects sensor values from the endpoints, such a curation can be called as \emph{``Advanced Machine Based Curation''}. If the user wants to take over the system then, it becomes more of a \emph{``User Based Curation''}. On top of all, the proposed system design is a pub-sub system. The Monitoring hardware is the publisher and Mobile based devices are subscribers. The cloud based server act as \emph{``brokers''}. Before any curation to work the device must initially register to the cloud server by their respective device tokens thereby, the broker service can later on perform push services. For handling the registration operation, the server have provided two different registration APIs --- one for the Android/iOS/Windows device using which mobile device can register itself and also subscribe to a particular monitoring device. And the other for the Monitoring hardware device in order for registering to server with its Geo-Location.

\subsection*{Machine Based Curation}

For machine based curation, we have made use of Centralized Cloud Based server - ``Parse''~\footnote{See www.parse.com/docs} for our core functionalities. This server is not only gives a hand for machine based curation but also, in providing various push based services to the Mobile Device(Android/iOS/Windows). The monitoring hardware used for this system is .Net Gadgeteer~\footnote{See www.netmf.com/gadgeteer/}. As said stated in main section, machine based curation works in modes --- \emph{Direct} and \emph{Advanced}. In case of Direct, the cloud server acts as a \emph{``broker''} between the Mobile device and the Monitoring Hardware. Whereas in case of Advanced, it is more of a hybrid model where both weather forecast information provided by the server and sensor data collected by the device is considered before performing an event. Sometimes is more of an \emph{``ìntelligent''} system.

\subsection*{User Based Curation}


\section*{Evaluation}

%Add evaluation here...

\section*{Inference}

%Add inference here...

\section*{Acknowledgment}

The authors would like to thank Florian Müller of Tele-Kooperation(TK) from Informatik department of Technische Universität Darmstadt for providing the opportunity to work in the selected topic.


%Bibliography....
\begin{thebibliography}{1}

\bibitem{ref001}
Kidd, CoryD. et,al. \emph{``The Aware Home: A Living Laboratory for Ubiquitous Computing Research''}.\hskip 1em plus
  0.5em minus 0.4em\relax  Springer Berlin Heidelberg (1999).

\bibitem{ref002}
Antonio J. Jara, Miguel A. Zamora, and Antonio F. Skarmeta.  \emph{``An internet of things---based personal device for diabetes therapy management in ambient assisted living (AAL).''}.\hskip 1em plus
  0.5em minus 0.4em\relax Personal Ubiquitous Comput. 15, 4 (April 2011), 431-440.  

\end{thebibliography}
\end{document}
